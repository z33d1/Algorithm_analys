	\newpage	
	\center{\section{Постановка задачи}}
	
	
	\flushleft{Реализовать алгоритм поиска расстояния Левенштейна}:
	\begin{enumerate}
		\item Базовый
		\item Модифицированный
		\item Рекурсивный
	\end{enumerate}

	\flushleft{Описание алгоритма.}
	\flushleft{Расстояние Левенштейна (также редакционное расстояние или дистанция редактирования) между двумя строками в теории информации и компьютерной лингвистике — это минимальное количество операций вставки одного символа, удаления одного символа и замены одного символа на другой, необходимых для превращения одной строки в другую.}



	\newpage
	\center{\section{Блок-схемы}}
	

	% 2 изображения в 1 строке
	\begin{figure}[h!]	
		\begin{minipage}[h]{0.25\linewidth}
			\center{\includegraphics[width=\linewidth]{block_1_1}}
			\caption{Блок-схема первой части матричного базового алгоритма}
		\end{minipage}
		\hfill
		\begin{minipage}[h]{0.7\linewidth}
			\center{\includegraphics[width=\linewidth]{block_1_2}}
			\caption{Блок-схема второй части матричного базового алгоритма}
		\end{minipage}
	\end{figure}

	\begin{figure}[hp]
		\center{\includegraphics[height=1.4\linewidth]{block_1_1}}
		\caption{Блок-схема рекурсивного алгоритма}
	\end{figure}



	\newpage
	\center{\section{Листинг}}
	

	\lstset{breaklines=true, numbers=left,
		 keywordstyle=\color{blue}, commentstyle=\color{red}}
	\lstinputlisting[language=Python]{../Levenstein.py}
	
	
	
	\newpage
	\center{\section{Тесты}}
	

	% Таблица
	\begin{center}
		\begin{tabular}{|c|c|c|c|}
			\hline
			Входные данные & Результат & Ожидаемый результат& Тест пройден\\
			
			\hline
			\makecell{s1 = "Привет" \\ s2 = "Приве" \\ // Пропущена одна буква} &
			\makecell{Базовый: 1\\Модифицированный: 1\\Рекурсивный: 1} & 
			\makecell{Базовый: 1\\Модифицированный: 1\\Рекурсивный: 1} & Да\\
			
			\hline
			\makecell{s1 = "Привет" \\ s2 = "Приветт" \\ //Добавлена лишняя буква} &
			\makecell{Базовый: 1\\Модифицированный: 1\\Рекурсивный: 1} & 
			\makecell{Базовый: 1\\Модифицированный: 1\\Рекурсивный: 1} & Да\\
			
			\hline
			\makecell{s1 = "Привет" \\ s2 = "Превет" \\ //Одна из букв изменена} &
			\makecell{Базовый: 1\\Модифицированный: 1\\Рекурсивный: 1} & 
			\makecell{Базовый: 1\\Модифицированный: 1\\Рекурсивный: 1} & Да\\
			
			\hline
			\makecell{s1 = "Привет" \\ s2 = "Првиет" \\ //Обмен местами 2 букв} &
			\makecell{Базовый: 2\\Модифицированный: 1\\Рекурсивный: 2} & 
			\makecell{Базовый: 2\\Модифицированный: 1\\Рекурсивный: 2} & Да\\
			
			\hline
			\makecell{s1 = "Мобильник" \\ s2 = "Моиблнег" \\ //Комбинированный случай} &
			\makecell{Базовый: 5\\Модифицированный: 4\\Рекурсивный: 5} & 
			\makecell{Базовый: 5\\Модифицированный: 4\\Рекурсивный: 5} & Да\\
			
			\hline
			\makecell{s1 = "\textbackslash 0"  \\ s2 = "\textbackslash 0" \\ //Пустые строки} &
			\makecell{Базовый: 0\\Модифицированный: 0\\Рекурсивный: 0} & 
			\makecell{Базовый: 0\\Модифицированный: 0\\Рекурсивный: 0} & Да\\
			
			\hline			
		\end{tabular}
	\end{center}	



	\newpage
	\center{\section{Замеры времени}}
	
	
	\begin{center}
		\begin{tabular}{|c|c|c|c|c|}
			\hline
			Входные данные & Базовый & Модифицированный & Рекурсивный\\
			
			\hline
			\makecell{s1 = "Привет" \\ s2 = "Приве" \\ // Пропущена одна буква} &
			0.00561 & 0.00331 & 0.18462\\
			
			\hline
			\makecell{s1 = "Привет" \\ s2 = "Приветт" \\ //Добавлена лишняя буква} &
			0.00574 & 0.00489 & 0.95220\\
			
			\hline
			\makecell{s1 = "Привет" \\ s2 = "Превет" \\ //Одна из букв изменена} &
			0.00536 & 0.00368 & 0.42712\\
			
			\hline
			\makecell{s1 = "Привет" \\ s2 = "Првиет" \\ //Обмен местами 2 букв} &
			0.00389 & 0.00403 & 0.43738\\
			
			\hline
			\makecell{s1 = "Мобильник" \\ s2 = "Моиблнег" \\ //Комбинированный случай} &
			0.00682 & 0.00713 & 2.93727\\
			
			\hline	
		\end{tabular}
	\end{center}	
	
	\center{\textsc{Замеры времени в секундах (среднее из 50 замеров)\\}}
		
	\flushleft{Графики в данной задаче не дадут отчетливую картину зависимости времени от длинны слов, так как время зависит так же от степени отличия слов, и способа отличия (перестановка букв стоящих рядом, или вставка и удаление буквы).}
	
	
	
	\newpage
	\center{\section{Выводы}}
	
	
	В резултате проведенных испытаний алгоритма было установлено, что:
	\begin{enumerate}
		\item Модифицированный алгоритм даст меньший результат, если пара соседних букв переставлены местами, однако мы потеряем во времени из-за большего количества сравнений
		\item Рекурсивный алгоритм всегда будет давать такой же результат, как базовый матричный. Однако рекурсивный алгоритм в несколько раз дольше, чем матричный.
	\end{enumerate}
	
	
	
	\newpage
	\center{\section{Заключение}}
	
	
	\flushleft{В ходе лабораторной работы были реализованы 3 алгоритма поиска расстояния Левенштейна: базовый и модифицированный матричные алгоритмы, и рекурсивный. Были получены навыки работы с матрицами и рекурсиями в Python, а так же работе с \LaTeX.}